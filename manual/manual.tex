\documentclass[a4paper]{article}


\usepackage{geometry}
\geometry{a4paper,margin=15mm,bindingoffset=15mm,heightrounded}


\usepackage{listings}
\lstloadlanguages{Ruby,C,bash}
\lstset{commentstyle=\itshape,stringstyle=\bfseries}


\newcommand{\autoc}{\emph{AutoC}}


\newcommand{\R}{\emph{Ruby}}


\newcommand{\C}{\emph{C}}


\newcommand{\Cpp}{\emph{C++}}


\lstnewenvironment{rs}{\lstset{language=Ruby,frame=tlbr,basicstyle=\small}}{}


\lstnewenvironment{cs}{\lstset{language=C,frame=tlbr,basicstyle=\small}}{}


\lstnewenvironment{bs}{\lstset{language=bash,frame=tlbr,basicstyle=\small}}{}



\begin{document}


\title{\autoc\ user's manual}
\author{Oleg A. Khlybov}


\maketitle


\section{Introduction}


\autoc\ is a host of \R\ modules related to automatic \C\ source code generation.


\newcommand{\codebuilder}{\emph{CodeBuilder}}
\section{\codebuilder\ module}


This module contains a facility to help generate \C\ modules.


\newcommand{\datastructbuilder}{\emph{DataStructBuilder}}
\section{\datastructbuilder\ module}


This module contains generators which create strongly-typed \C\ data containers similar to those found in \Cpp\ \emph{STL} library: \emph{vectors}, \emph{lists}, \emph{sets}, \emph{maps}, and more.


\subsection{Quick start}


Suppose one wants to have a \C\ set container which operates on \lstinline[language=C++]{int} data type.
In \Cpp\ world this would be a \lstinline[language=C++]{std::set<int>} type (with the difference that \lstinline[language=C++]{std::set<>} is a tree-based set while \autoc\ currently contains generator for a hash-based set implementation only).


\begin{enumerate}


\item Create a \R\ definition file \emph{containers.rb}:


\begin{rs}
require "autoc/data_struct_builder"

Int = {:type=>"int"}

CodeBuilder::CModule.generate!("containers") do |m|
	m << DataStructBuilder::HashSet.new("IntSet", Int)
end
\end{rs}


The above file contains \R\ code which instructs to generate the set data structure \emph{IntSet} and put generated \C\ code into specific \C\ compilation unit named \emph{containers}.



\item Generate \C\ source files:


\lstinline[language=bash]!> ruby containers.rb!


Provided that \autoc\ files are somewhere in the \R\ path, the above command will output two \C\ files: the \C\ module interface file \emph{containers\_auto.h} and the \C\ module implementation file \emph{containers\_auto.c}.


\item Create main \C\ source file \emph{main.c}:


\begin{cs}
#include <stdio.h>

#include "containers_auto.h"

int main(int argc, char** argv) {
	IntSet set;
	IntSetCtor(&set, 16);
	IntSetPut(&set, 0);
	IntSetPut(&set, 1);
	IntSetPut(&set, 0);
	printf("size(set)=%d\n", IntSetSize(&set));
	IntSetDtor(&set);
	return 0;
}
\end{cs}


The code above creates a set container \emph{set} on stack, puts three integer values into it, prints the number of elements the set actually contains and finally destroys the container.


\item Build the executable:


\lstinline[language=bash]!> cc -o runme main.c containers_auto.c!


On successful execution the above command will produce the executable \emph{runme} with defined logic.
Correct output for this program would be 
\begin{verbatim}
size(set)=2
\end{verbatim}
since the set will reject inclusion of duplicate values (in the above case it's \emph{0}).

\end{enumerate}


\end{document}